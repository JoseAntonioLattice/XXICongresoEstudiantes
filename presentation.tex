\documentclass[11pt]{beamer}
\usetheme{Bergen}
\usepackage[utf8]{inputenc}
\usepackage{amsmath}
\usepackage{amsfonts}
\usepackage{amssymb}
\usepackage{graphicx}
\usepackage{hyperref}
\usepackage{fontawesome}
\usepackage{xcolor}
\usepackage{movie15}
\usepackage{dsfont}
\usepackage{tikz}
\def\insertauthorindicator{MSc}
\def\insertinstituteindicator{Institute}
\def\insertdateindicator{Date}
\usetikzlibrary{decorations.markings}

\tikzset{
  invisible/.style={opacity=0},
  visible on/.style={alt={#1{}{invisible}}},
  alt/.code args={<#1>#2#3}{%
    \alt<#1>{\pgfkeysalso{#2}}{\pgfkeysalso{#3}} % \pgfkeysalso doesn't change the path
  },
}
\setbeamertemplate{footline}[frame number]
\author{José Antonio García Hernández}
\title{\huge Introduction to Lattice QCD}
\setbeamercovered{transparent} 
\setbeamertemplate{navigation symbols}{} 

\institute{Instituto de Ciencias Nucleares, UNAM} 
\date{5 diciembre 2025 \\ XXI Congreso de Estudiantes del PCF} 
\subject{Lattice Field Theories} 
\begin{document}

\begin{frame}
\titlepage

\centering
\includegraphics[width=5cm]{LatticeQCD.png}
\end{frame}

%\begin{frame}
%\tableofcontents
%\end{frame}

\section{Introduction}
\subsection{Functional Integral}
\begin{frame}{Functional integral formalism}

    Sum over ALL field configurations $[\phi(x)]$
    $$Z = \int \mathcal{D}\phi \, e^{-S[\phi]},$$
    where $S[\phi]$ is the Euclidean action% de la configuración
    $$ S[\phi] = \int d^4 x \, \left( \frac{1}{2}\partial_{\mu}\phi \partial_{\mu}\phi + V(\phi)\right).$$
    $n$-point function
    $$ \langle O[\phi(x_1)] \cdots \rangle =  \frac{1}{Z}\int \mathcal{D}\phi \, O[\phi(x_1)]\cdots e^{-S[\phi]}.$$
\end{frame}

\subsection{Lattice regularization}
\begin{frame}{Lattice Regularization}
    UV divergencies.\\~
    
    Discretization of the lattice: %Discretizar el espaciotiempo
    
    \begin{center}
    \begin{tikzpicture}[scale = 0.5]
    	\draw (-0.5,-0.5) grid (2.5,2.5);
    	\fill[red] (1,1) circle(0.15);
    	\draw[red] (1.5,1.5) node[]{$\phi_x$};
    	\draw[blue] (2,0) node[above right]{ $a = 1$};
    	\draw[-latex, thick] (4,1) --(4,2) node[left]{\small $\hat{\nu}$};
	\draw[-latex, thick] (4,1) --(5,1) node[right]{\small $\hat{\mu}$};
    \end{tikzpicture}
    \end{center}
    \begin{itemize}
    		\item $\phi(x) \to \phi_x$
         \item $\displaystyle\partial_\mu\phi(x) \to \phi_{x + \hat{\mu}} - \phi_x$
         \item $\displaystyle\int d^4x \to \sum_{x}$
    \end{itemize}
    
   
    NON-PERTURBATIVE formalism.
\end{frame}

\begin{frame}{Markov chains}

        Markov chain: A chain of configurations where the current configuration only depends on the previous one.

    %Generamos una cadena de configuraciones donde la configuración actual depende solo de la anterior (cadena de Markov)
    
    $$ [\phi_1] \to [\phi_2] \to \cdots$$

    $$\langle O \rangle = \frac{1}{N}\sum_{i=1}^N O[\phi_i]$$
    where $[\phi_i]$ is generated with probability $\frac{1}{Z}e^{-S[\phi_i]}$.
\end{frame}

\begin{frame}{Importance sampling}
    In order to generate good configurations the algorithm needs to satisfy   
    %Para generar configuraciones, el algoritmo necesita cumplir
    \begin{itemize}
        \item<1-> Ergodicity
        \begin{center}
        		\begin{tikzpicture}
        			\draw (0,0) rectangle (4,2);
        			\draw[blue] (0.1,0.5) -- (1,1.18) -- (3, 0.3) -- (3.7, 1.9) -- (1, 0.1) --(0.8,1.8) -- (2,1.5);
        		\end{tikzpicture}
        \end{center}
        \item<2-> Detailed balance
        \begin{center}
        \begin{tikzpicture}
    		\coordinate (A) at (0.7,0.7);
    		\coordinate (B) at (3,1.5);
    		\draw (0,0) rectangle (4,2);
    		\fill[] (A) circle(0.1) node[left]{$A$};
    		\fill[] (B) circle(0.1) node[right]{$B$};
    		\draw[blue,thick, -latex] (A) to[out = 90, in = 180] (B);
    	\draw[red,thick, -latex] (B) to[out = -90, in = 0] (A);		
    		
    \end{tikzpicture}
    \end{center}
    $$ p[A]p[A \to B] = p[B]p[B \to A]$$
    \end{itemize}
    
    
\end{frame}

\begin{frame}{Update algorithms}
    Local
    \begin{itemize}
        \item Metropolis
        \item Glauber
        \item Heatbath
        %\item Hybrid Monte Carlo
        \item etc\dots
    \end{itemize}
    
    \ \\~
    
    Global
    \begin{itemize}
        \item Wolff
        \item Swendsen-Wang
        \item Hybrid Monte Carlo (HMC)
    \end{itemize}
\end{frame}

\begin{frame}{Monte Carlo simulation}

    \begin{itemize}
    		\item Initial configuration \emph{\textcolor{red}{hot}} or \emph{\textcolor{blue}{cold}} \emph{start}.
        \item Generation of configurations.
        \item Thermalization: attain equilibrium.
        \item Taking of measurements.
        \item Computation of averages and errors (\emph{jackknife} commonly used).
        \item Interpretation of results.
    \end{itemize}

\end{frame}

\begin{frame}{Lattice gauge field theory}
Compact formulation (NO gauge fixing): $$A_{x,\mu} \in \mathfrak{su}(N) \to U_{x,\mu} \in \text{SU}(N).$$
$$\text{Link variable: }U_{x,\mu} = e^{iA_{x,\mu}}.$$ 

	\begin{tikzpicture}[scale = 0.7,decoration={markings, mark= at position 0.6 with {\arrow{latex}}}]
	
		\draw[step=2cm,color=gray] (-1,-1) grid (3,3);
		
		\draw[postaction={decorate},very thick,red] (0,0) --(2,0);
		\draw[postaction={decorate},very thick,red] (2,0)--(2,2);
		\draw[postaction={decorate},very thick,red] (2,2)--(0,2);
		\draw[postaction={decorate},very thick,red] (0,2)--(0,0);
		
		\filldraw[black] (0,0) circle(0.1);
		\filldraw[black] (2,0) circle(0.1);
		\filldraw[black] (2,2) circle(0.1);
		\filldraw[black] (0,2) circle(0.1);
		\draw[] (0,0) node[below left] {\tiny  $x$};
		\draw[] (2,0) node[below right] {\tiny  $x+\hat\mu$};
		\draw[] (0,2) node[above left] {\tiny  $x+\hat\nu$};
		\draw[] (2,2) node[above right] {\tiny  $x+\hat\mu+\hat\nu$};
		
		\draw[] (8,2) node[] {plaquette:};
		
		\draw[] (8,1) node[] {\small  $U_{x,\mu\nu} = U^{\dagger}_{x,\nu} U^{\dagger}_{x+\hat\nu,\mu}U_{x+\hat\mu,\nu}U_{x,\mu}$};
		
		\draw[blue] (1,0) node[below] {\small $U_{x,\mu}$};
		\draw[blue] (2,1) node[right] {\small  $U_{x+\hat\mu,\nu}$};	
		\draw[blue] (1,2) node[above] {\small  $U^{\dagger}_{x+\hat\nu,\mu}$};
		\draw[blue] (0,1) node[left] {\small  $U^{\dagger}_{x,\nu}$};
	\end{tikzpicture}
     
    Wilson's standard action
$$ S[U] = \frac{\beta}{N}\sum_{x}\sum_{\mu < \nu} \text{Tr Re} \left[\mathds{1} - U_{x,\mu\nu} \right], $$
where $\beta = 2N/g^2$, with $g$ the gauge coupling. 
\end{frame}

\begin{frame}{Fermions on the lattice}
    Fermi stastistics implemented by \textit{anticommuting} Grassman variables
    $$ \psi_i \psi_j = - \psi_j\psi_i, \ \ \ i,j\in\{1,\dots,N\}.$$
    
    Free fermion functional integral (fermion determinant)
    $$Z_F = \int D\bar{\Psi} D\Psi\, \exp(-\bar\Psi D_{\text{Dirac}}\Psi) = \det D_{\text{Dirac}},$$
    where $D_{\text{Dirac}}$ is the Dirac operator.

\end{frame}

\begin{frame}

Fermion fields $\psi$ satisfy anti-periodic boundary conditions in the Euclidean time direction and periodic elsewhere. \\~

Computation of fermion determinant very inefficient. \\~

Better to invert the Dirac matrix.\\~

On the lattice, instead of working with fermions one works with \textit{pseudo-fermions}. \\~

Pseudo-fermions are bosonic variables with a similar partition function to the  fermion determinant for two degenerate mass flavors.

\end{frame}

\begin{frame}{Lattice QCD}

    In four spacetime dimensions and two fermion flavors the Dirac Matrix of QCD has     
    $$ (12L_t L_xL_yL_z)^2$$
    entries.  \\~
    
For a square matrix with $L_t= L_x = L_y = L_z = 32$ the matrix has $(1.3\times 10^7)^2$ elements. \\~
    
    We have to do several matrix operations for each HMC update.\\~
    
    Very difficult to simulate.

\end{frame}

\subsection{2d Schwinger model}
\begin{frame}{Example: Schwinger model}
Lagrangian of the Schwinger model 
$$\mathcal{L} = \frac{1}{4}F_{\mu\nu}F_{\mu\nu} + \sum_f\bar\psi^f(\gamma_{\mu}D_{\mu}+m_0)\psi^f,$$
where $F_{\mu\nu} = \partial_{\mu} A_{\nu} - \partial_{\nu} A_{\mu}$ and $D_{\mu} = \partial_{\mu} + ieA_{\mu}$.

Action on the lattice
\begin{eqnarray*}
    S[\psi,\bar\psi,U]& = & \beta\sum_{x}\text{Re}(1-U_{x,12}) + \\
    & &\sum_{x,f}\bar{\psi}^f_x\sum_{\mu=1}^2\gamma_{\mu}\frac{U_{x,\mu}\psi^{f}_{x+\hat{\mu}}-U_{x,-\mu}\psi^{f}_{x-\hat{\mu}}}{2} + \\
    & &\sum_{x,f,\mu}m_0\bar\psi^f_x\psi^f_x + \text{other terms},
\end{eqnarray*}
where $\beta = 1/e^2$.
\end{frame}

\begin{frame}

In 1+1 dimensions

$$\gamma_1 = \begin{pmatrix}
0 & 1\\
1 & 0
\end{pmatrix}, \ \ \ \gamma_2 = \begin{pmatrix}
0 & -i\\
i & 0
\end{pmatrix}.
$$
and $\gamma_5 = \begin{pmatrix}
-1 & 0\\
0 & 1
\end{pmatrix}$.
\end{frame}

\begin{frame}{Plaquette value}
One of the simplest observables to measure is the sum of plaquette variables
$$\text{plaquette value} = \left\langle\frac{1}{V}\sum_{x}\text{Re }U_{x,12}\right\rangle.$$
\end{frame}

\begin{frame}{Plaquette value}
\begin{center}
\includegraphics[scale=0.7]{plaquette.pdf}
\end{center}
\end{frame}


\begin{frame}{Pion correlation function}
   Pion propagator $C_{\pi}(t)=\left\langle O_n\bar O_m\right\rangle $, where $O_n = \bar d_n \gamma_5 u_n$ and $t = |n-m|$.
   $$C_{\pi}(t) = \left\langle\sum_{\alpha,\beta}\left | D^{-1}_{\text{Dirac}}(n|m)_{\alpha\beta}\right|^2\right\rangle.$$  
   
   Decays exponentially in Euclidean time
   $$ C_{\pi}(t)\propto \exp(-m_{\pi}t).$$
   
   To take into account periodicity we use
    $$C_{\pi}(t)\propto \cosh(m_{\pi}(t-L_t/2)).$$
\end{frame}


\begin{frame}{Pion correlation function}
\centering
\includegraphics[scale=0.9]{correlation_pion.pdf}
%\include{correlation_pion.tex}
\end{frame}


\begin{frame}{Topological susceptibility with slab method}
    
    Divide the periodic volume $V$ into sub-volumes or \textit{slabs} of size $xV$, where $x\in[0,1]$. \\~
    
    Extract $\chi_t$ from the fluctuations of the topological charge $q$ within these slabs.\\~
    
    At fixed topological charge $Q$
    
    $$p(q)p(Q-q) \propto \exp\left(-\frac{1}{2\chi_tV}\frac{(q-xQ)^2}{x(1-x)} \right),$$
    from which we infer
    $$\langle q^2\rangle = \chi_tVx(1-x) + Q^2x^2. $$
\end{frame}

\begin{frame}{Topological susceptibility with slab method}
    \centering
    \includegraphics[scale=0.7]{slab.pdf}
   
\end{frame}


\section{Summary}
\begin{frame}{Summary}

Lattice regularization is a non-perturbative and gauge invariant approach.\\~

Compact formulation does NOT require gauge fixing. \\~

Lattice formulation very robust.\\~

QCD very difficult on the lattice. \\~

Schwinger model: toy model for QCD. \\~

The slab method efficient when topological freezing occurs.
    
\end{frame}

\begin{frame}{References}

\textbf{¡Follow me on \faGithub !}
\url{https://github.com/JoseAntonioLattice}\\~

\textbf{Lattice QCD books}

Gattringer C., Lang C.B., Quantum Chromodynamics on the Lattice: An Introductory
Presentation. \\~

Knechtli F., Günther M., Peardon M., Lattice Quantum Chromodynamics Practical Essentials. \\~

\textbf{The slab method}

Bietenholz W., de Forcrand P., Gerber U., Topological Susceptibility from Slabs. JHEP 12 (2015).
 
\end{frame}


\end{document}
