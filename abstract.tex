\documentclass[12pt,a4paper]{article}
\usepackage[utf8]{inputenc}
\usepackage{amsmath}
\usepackage{amsfonts}
\usepackage{amssymb}
\usepackage{graphicx}
\author{José Antonio García Hernández}
\title{Introduction to Lattice QCD}
\begin{document}
\maketitle

The theory that describes the behavior of quarks and gluons is known as Quantum Chromodynamics (QCD). The lattice regularization, together with the compact formulation of gauge fields, has been proved successful in studying the non-perturbative regime, where analytical methods fail. Lattice QCD has become a great tool in particle physics, although remains numerically challenging.

In this talk, I present an introductory overview covering the essential concepts of lattice methods. As an example, I show results for the two-dimensional Schwinger  model with two degenerate quark flavors. We measured the topological charge of the  gauge field, the expectation value of the plaquette, and the mass of the lightest  hadron in this theory, which corresponds to the pion in QCD.


\end{document}
